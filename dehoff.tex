\documentclass[a4paper, 12pt]{article}
\usepackage{amsmath, amsthm}
\usepackage{geometry}

\newcommand{\Left}{\mathopen{}\mathclose\bgroup\left}
\newcommand{\Right}{\aftergroup\egroup\right}

\theoremstyle{definition}
\newtheorem{problem}{Problem}[section]

\theoremstyle{remark}
\newtheorem*{remark}{Remark}
\newtheorem*{answer}{Answer}

\renewcommand{\theenumi}{\alph{enumi}}

\title{DeHoff's Thermodynamics}
\author{Chen-Pang He}

\begin{document}
\maketitle
This document contains unofficial answers to homework problems in Robert
DeHoff's \textit{Thermodynamics in Materials Science}, 2nd ed.

\section{Why study thermodynamics?}
No homework problems are in this chapter.

\section{The structure of thermodynamics}
\begin{problem}
    Classify the following thermodynamic systems in the five categories defined
    in Section 2.1:
    \begin{enumerate}
        \item A solid bar of copper.
        \item A glass of ice water.
        \item A yttrium stabilized zirconia furnace.
        \item A Styrofoam coffee cup.
        \item A eutectic alloy turbine blade rotating at 20,000 r/min.
    \end{enumerate}

    You may find it necessary to qualify your answer by defining the system
    more precisely; state your assumptions.
\end{problem}

\begin{remark}
    \renewcommand{\theenumi}{\arabic{enumi}}
    The aforementioned five categories are:
    \begin{enumerate}
        \item Unary vs.\ multicomponent
        \item Homogeneous vs.\ heterogeneous
        \item Closed vs.\ open
        \item Nonreacting vs.\ reacting
        \item Otherwise simple vs.\ complex
    \end{enumerate}
\end{remark}

\begin{answer}
    TODO
\end{answer}

\begin{problem}
\end{problem}

\begin{problem}
    Determine which of the following properties of a thermodynamic system are
    extensive properties and which are intensive:
    \begin{enumerate}
        \item The mass density.
        \item The molar density.
        \item The number of gram atoms of aluminum in a chunk of alumina.
        \item The potential energy of a system in a gravitational field.
        \item The molar concentration of NaCl in a salt solution.
        \item The heat absorbed by the gas in a cylinder when it is compressed.
    \end{enumerate}
\end{problem}

\begin{answer}
    (Note that \textbf{densities} convert extensive properties into
    intensive ones.)
    \begin{enumerate}
        \item Intensive
        \item Intensive
        \item Extensive
        \item Extensive
        \item Intensive
        \item Extensive
    \end{enumerate}
\end{answer}

\begin{problem}
    Why is heat a process variable?
\end{problem}

\begin{problem}
    Write the total differential of the function:
    \[ z = 12 u^3 v \cos(x) \]

    \begin{enumerate}
        \item Identify the coefficients of the three differentials in this
            expression as appropriate partial derivatives.
        \item Show that three Maxwell relations hold among these coefficients.
    \end{enumerate}
\end{problem}

\begin{answer}
    \begin{enumerate}
        \item Total derivative can be constructed from partial derivatives.
            \begin{align*}
                \frac{\partial z}{\partial u} &= 36 u^2 v \cos(x) \\
                \frac{\partial z}{\partial v} &= 12 u^3 \cos(x) \\
                \frac{\partial z}{\partial x} &= -12 u^3 v \sin(x)
            \end{align*}

            \[ dz = 36 u^2 v \cos(x) \,du
                + 12 u^3 \cos(x) \,dv
                - 12 u^3 v \sin(x) \,dx.\]

        \item Maxwell relations are direct corollaries of Schwarz's theorem.
            \[
                \begin{gathered}
                    \frac\partial{\partial v}\frac{\partial z}{\partial u}
                        = 36 u^2 \cos(x)
                        = \frac\partial{\partial u}\frac{\partial z}{\partial v} \\
                    \frac\partial{\partial x}\frac{\partial z}{\partial u}
                        = -36 u^2 v \sin(x)
                        = \frac\partial{\partial u}\frac{\partial z}{\partial x} \\
                    \frac\partial{\partial x}\frac{\partial z}{\partial v}
                        = -12 u^3 \sin(x)
                        = \frac\partial{\partial v}\frac{\partial z}{\partial x} \\
                \end{gathered}
            \]
    \end{enumerate}
\end{answer}

\begin{problem}
\end{problem}

\section{The laws of thermodynamics}

\section{Thermodynamic variables and relations}
\begin{problem}
    Write out the combined statements of the first and second laws for the
    energy functions, $U = U(S, V)$, $H = H(S, P)$, $F = F(T, V)$ and $G = G(T,
    P)$.  Assume $\delta W'$ is zero:
    \begin{enumerate}
        \item Write out all eight coefficient relations.
        \item Derive all four Maxwell relations.
    \end{enumerate}
    for these equations.
\end{problem}

\begin{answer}
    \begin{enumerate}
        \item According to definitions,
            \begin{align*}
                H &= U + PV \\
                F &= U - TS \\
                G &= H - TS.
            \end{align*}

            By the first law,
            \[ \Delta U = Q + W \]
            \[ W(S, V) = -\int PdV \]
            \begin{equation}
                \Delta U(S, V) = Q -\int PdV. \label{eq:1st}
            \end{equation}

            By the second law,
            \[ \Delta S = \int dS = \int \frac{\delta Q}{T} \]
            \begin{equation}
                Q = \int \delta Q = \int TdS. \label{eq:2nd}
            \end{equation}

            Combining equations \eqref{eq:1st} and \eqref{eq:2nd},
            \[ \Delta U = \int dU = \int TdS - \int PdV \]
            \begin{equation}
                dU = TdS - PdV. \label{eq:dU}
            \end{equation}

            The other differentials are computed with \eqref{eq:dU}.
            \begin{align*}
                dH &= TdS + VdP \\
                dF &= -SdT - PdV \\
                dG &= -SdT + VdP.
            \end{align*}

        \item Maxwell relations can be derived with Schwarz's theorem.
            \[
                \frac{\partial^2 U}{\partial S \partial V}
                = \left( \frac{\partial T}{\partial V} \right)_S
                = -\left( \frac{\partial P}{\partial S} \right)_V
            \]
            \[
                \frac{\partial^2 H}{\partial S \partial P}
                = \left( \frac{\partial T}{\partial V} \right)_S
                = \left( \frac{\partial V}{\partial S} \right)_P
            \]
            \[
                -\frac{\partial^2 F}{\partial T \partial V}
                = \left( \frac{\partial S}{\partial V} \right)_T
                = \left( \frac{\partial P}{\partial S} \right)_V
            \]
            \[
                \frac{\partial^2 G}{\partial S \partial P}
                = -\left( \frac{\partial S}{\partial V} \right)_T
                = \left( \frac{\partial V}{\partial S} \right)_P
            \]
    \end{enumerate}
\end{answer}

\begin{problem}
    Derive the ratio relation Equation 4.30:
    \[
        \left( \frac{\partial Z}{\partial X} \right)_Y
        \left( \frac{\partial X}{\partial Y} \right)_Z
        \left( \frac{\partial Y}{\partial Z} \right)_X
        = -1.
    \]
\end{problem}

\begin{answer}
    Let $F$ be a function satisfying
    \[ F(X, Y, Z) = 0.\]

    Applicability of inverse function theorem is assumed in thermodynamic
    setting: $F$ is deemed differentiable and its partial derivatives are
    nonzero.  For brevity, let
    \begin{align*}
        A &= \frac{\partial F}{\partial X} \\
        B &= \frac{\partial F}{\partial Y} \\
        C &= \frac{\partial F}{\partial Z}.
    \end{align*}

    Then the total differential of $F$ is
    \[ dF = AdX + BdY + CdZ = 0.\]

    The needed partial derivatives can be reconstructed from the total
    differential.
    \begin{align*}
        -CdZ &= AdX + BdY \\
        -AdX &= BdY + CdZ \\
        -BdY &= CdZ + AdX
    \end{align*}

    \[
        \left( \frac{\partial Z}{\partial X} \right)_Y
        \left( \frac{\partial X}{\partial Y} \right)_Z
        \left( \frac{\partial Y}{\partial Z} \right)_X
        = \left( -\frac AC \right)
          \left( -\frac BA \right)
          \left( -\frac CB \right)
        = -1.
    \]
\end{answer}

\begin{problem}
\end{problem}
\end{document}
