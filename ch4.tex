\begin{@empty}
\section{Thermodynamic variables and relations}
\begin{problem}
    Write out the combined statements of the first and second laws for the
    energy functions, $U = U(S, V)$, $H = H(S, P)$, $F = F(T, V)$ and $G = G(T,
    P)$.  Assume $\delta W'$ is zero:
    \begin{enumerate}
        \item Write out all eight coefficient relations.
        \item Derive all four Maxwell relations.
    \end{enumerate}
    for these equations.
\end{problem}

\begin{answer}
    \begin{enumerate}
        \item According to definitions,
            \begin{align*}
                H &= U + PV \\
                F &= U - TS \\
                G &= H - TS.
            \end{align*}

            By the first law,
            \[ \Delta U = Q + W \]
            \[ W(S, V) = -\int PdV \]
            \begin{equation}
                \Delta U(S, V) = Q -\int PdV. \label{eq:1st}
            \end{equation}

            By the second law,
            \[ \Delta S = \int dS = \int \frac{\delta Q}{T} \]
            \begin{equation}
                Q = \int \delta Q = \int TdS. \label{eq:2nd}
            \end{equation}

            Combining equations \eqref{eq:1st} and \eqref{eq:2nd},
            \[ \Delta U = \int dU = \int TdS - \int PdV \]
            \begin{equation}
                dU = TdS - PdV. \label{eq:dU}
            \end{equation}

            The other differentials are computed with \eqref{eq:dU}.
            \begin{align*}
                dH &= TdS + VdP \\
                dF &= -SdT - PdV \\
                dG &= -SdT + VdP.
            \end{align*}

        \item Maxwell relations can be derived with Schwarz's theorem.
            \[
                \frac{\partial^2 U}{\partial S \partial V}
                = \left( \frac{\partial T}{\partial V} \right)_S
                = -\left( \frac{\partial P}{\partial S} \right)_V
            \]
            \[
                \frac{\partial^2 H}{\partial S \partial P}
                = \left( \frac{\partial T}{\partial P} \right)_S
                = \left( \frac{\partial V}{\partial S} \right)_P
            \]
            \[
                -\frac{\partial^2 F}{\partial T \partial V}
                = \left( \frac{\partial S}{\partial V} \right)_T
                = \left( \frac{\partial P}{\partial T} \right)_V
            \]
            \[
                \frac{\partial^2 G}{\partial T \partial P}
                = -\left( \frac{\partial S}{\partial P} \right)_T
                = \left( \frac{\partial V}{\partial T} \right)_P
            \]
    \end{enumerate}
\end{answer}

\begin{problem}
    Derive the ratio relation Equation 4.30:
    \[
        \left( \frac{\partial Z}{\partial X} \right)_Y
        \left( \frac{\partial X}{\partial Y} \right)_Z
        \left( \frac{\partial Y}{\partial Z} \right)_X
        = -1.
    \]
\end{problem}

\begin{answer}
    Let $F$ be a function satisfying
    \[ F(X, Y, Z) = 0.\]

    Applicability of inverse function theorem is assumed in thermodynamic
    setting: $F$ is deemed differentiable and its partial derivatives are
    nonzero.  For brevity, let
    \begin{align*}
        A &= \frac{\partial F}{\partial X} \\
        B &= \frac{\partial F}{\partial Y} \\
        C &= \frac{\partial F}{\partial Z}.
    \end{align*}

    Then the total differential of $F$ is
    \[ dF = AdX + BdY + CdZ = 0.\]

    The needed partial derivatives can be reconstructed from the total
    differential.
    \begin{align*}
        -CdZ &= AdX + BdY \\
        -AdX &= BdY + CdZ \\
        -BdY &= CdZ + AdX
    \end{align*}

    \[
        \left( \frac{\partial Z}{\partial X} \right)_Y
        \left( \frac{\partial X}{\partial Y} \right)_Z
        \left( \frac{\partial Y}{\partial Z} \right)_X
        = \left( -\frac AC \right)
          \left( -\frac BA \right)
          \left( -\frac CB \right)
        = -1.
    \]
\end{answer}

\begin{problem}
    The molar volume of Al$_2$O$_3$ at 25\si\celsius\ and 1 atm is 25.715
    cc/mol.  Its coefficient of thermal expansion is \SI{26e-6}{\per\kelvin}
    and the coefficient of compressibility is \SI{8.0e-7}{\per\atm}.  Estimate
    the molar volume of Al$_2$O$_3$ at 400\si\celsius\ and 10 kbars pressure
    (\SI{10e3}{\atm}).
\end{problem}

\begin{answer}
    \begin{@empty}
        \newcommand{\thisV}{\SI{25.715}{cc/\mol}}
        \newcommand{\thisT}{25\si\celsius}
        \newcommand{\thisP}{\SI{1}{atm}}
        \newcommand{\thisa}{\SI{26e-6}{\per\kelvin}}
        \newcommand{\thisb}{\SI{8.0e-7}{\per\atm}}

        The molar volume $V$ is a function of temperature $T$ and pressure $P$.
        This is an initial value problem.
        \[ V(\thisT, \thisP) = \thisV \]
        \[ \frac1V \frac{\partial V}{\partial T} = \alpha = \thisa \]
        \[ -\frac1V \frac{\partial V}{\partial P} = \beta = \thisb.\]

        Convert the system of partial differential equations into differentials.
        \[ dV = V \alpha dT - V \beta dP.\]

        \newcommand{\varthisV}{\SI[per-mode=fraction]{25.715}{\cc\per\mol}}
        \newcommand{\exponent}{\thisa \left( T - \thisT \right)
            - \thisb \left( P - \thisP \right)}
        This is a separable equation, which is solvable by mere integration.
        \[
            \int_{\varthisV}^V \frac1V dV
            = \int_{\thisT}^T \thisa dT
            - \int_{\thisP}^P \thisb dP
        \]
        \[ \ln \frac{V}{\thisV} = \exponent \]
        \[ V = \varthisV \cdot \exp \Left( \exponent \Right).\]
    \end{@empty}

    Compute the volume in the final state with the just solved $V(T, P)$.
    \[ V(400\si\celsius, \SI{10e3}{\atm}) = \SI{25.760}{\cc/\mol}.\]
\end{answer}

\begin{problem}
\end{problem}

\begin{answer}
    \begin{enumerate}
        \item Entropy can be computed with little derivation from its
            definition.
            \[ \Delta S = \int \frac{\delta Q}{T} = \int \frac{C_P}{T} dT \]

            Isobaric specific heat capacity of selected elements can be found
            in Appendix B.
            \[ C_P = 11.17 + \num{37.78e-3}\,T + \frac{\num{3.18e5}}{T^2} \left( \si{\joule\per\mol\per\kelvin} \right) \]

            Heat capacity has the same dimension as entropy.
            \begin{align*}
                \Delta S &= \int_{300}^{1300} \left( \frac{11.17}{T} + \num{37.78e-3} + \frac{\num{3.18e5}}{T^3} \right) dT \\
                    &= \left[ 11.17 \ln T + \num{37.78e-3}\, T - \frac{\num{1.59e5}}{T^2} \right]_{300}^{1300} \\
                    &= 11.17 \ln \frac{1300}{300} + 37.78
                        + \num{1.59e5} \left( \frac{1}{300^2} - \frac{1}{1300^2} \right) \\
                    &\approx 55.83 \left( \si{\joule\per\mol\per\kelvin} \right)
            \end{align*}

        \item Computing isothermal entropy is somewhat tricky because
            involvement of a Maxwell relation.
            \[
                \left( \frac{\partial S}{\partial P} \right)_T
                = -\frac{\partial^2 G}{\partial T \partial P}
                = -\left( \frac{\partial V}{\partial T} \right)_P
                = -V \alpha
            \]
            \[ \Delta S = \int \left( \frac{\partial S}{\partial P} \right)_T dP = -\int V \alpha dP \]

            \begin{@empty}
                \newcommand{\stdT}{\SI{298}{\kelvin}}
                \newcommand{\stdP}{\SI{1}{\atm}}
                \newcommand{\thisa}{\SI{40e-6}{\per\kelvin}}
                \newcommand{\thisb}{\SI{26e-7}{\per\atm}}
                \newcommand{\thisVS}{\SI[per-mode=fraction]{6.60}{\cc\per\mol}}
                \newcommand{\exponent}{\num{80e-6} - \thisb \left( P - \stdP \right)}

                Solve $V(P)$ from the differential equation where $T =
                \SI{300}{\kelvin}$.
                \[ dV = V\alpha dT - V\beta dP.\]

                With only molar volume at \stdT, we still have to integrate
                from \stdT\ and \stdP.
                \[
                    \int_{V^{\text S}}^V \frac1V dV
                    = \int_{\stdT}^{\SI{300}{\kelvin}} \thisa dT
                    + \int_{\stdP}^P \thisb dP
                \]
                \[ V(\stdP) = V^{\text S} = \SI{6.60}{\cc/\mol} \]
                \[
                    V(P) = \thisVS \cdot \exp \Left( \exponent \Right).
                \]

                Now find entropy change with $V(P)$.  Note that the unadjusted
                equation would produce energy in \si{\atm.cc} due to units for
                $V^{\text S}$ and $\beta$.  Multiply the conversion factor to get
                energy in joules.
                \begin{center}
                    \SI{1}{\atm.cc} = \SI{0.101325}{\joule}
                \end{center}
                \[ \Delta S(P) = -\thisa \int_{\stdP}^P V(P)\,dP \]
                \begin{align*}
                    \int_{\stdP}^P V dP &= \thisVS \int_{\stdP}^P \exp \Left( \exponent \Right) dP \\
                    &= -\thisVS \cdot \frac{\exp \Left( \num{80e-6} \Right)}{\thisb}
                        \left[ \exp \Left( -\thisb \left( P - \stdP \right) \Right) \right]_{\stdP}^P \\
                    &= -\thisVS \cdot \frac{\exp \Left( \num{80e-6} \Right)}{\thisb}
                        \left( \exp \Left( -\thisb \left( P - \stdP \right) \Right) - 1 \right)
                \end{align*}
                \begin{align*}
                    \Delta S(\SI{100e3}{\atm}) &= \SI[per-mode=fraction]{0.101325}{\joule\per\atm\per\cc}
                        \cdot \thisa \cdot \thisVS \cdot \frac{\exp \Left( \num{80e-6} \Right)}{\thisb} \\
                        &\quad \left( \exp \Left( -\thisb \left( \SI{100e3}{\atm} - \stdP \right) \Right) - 1 \right) \\
                        &= \SI{-2.36}{\joule\per\mol\per\kelvin}
                \end{align*}
            \end{@empty}

            This answer departs from the published answer because the textbook
            author thoroughly assumes $\exp x \approx 1 + x$.  The answer
            provided here is more precise.
    \end{enumerate}
\end{answer}

\begin{problem}
\end{problem}

\begin{problem}
\end{problem}

\begin{problem}
\end{problem}

\begin{answer}
    Consider the differential of entropy:
    \[ dS = \frac{C_P}{T} dT - V\alpha dP.\]

    Obviously $C_P / T$ is a function of $T$
    \[ \frac{C_P}{T} = \frac{7R}{2T}.\]

    $V\alpha$ is unlikely to be a function of $P$ at the first glance, but it
    is.
    \[ V\alpha = \frac{RT}{P} \frac{1}{T} = \frac{R}{P}.\]

    Therefore, we can find entropy with integration.
    \[ S(T, P) - S_{298}^\circ = \int_{\SI{298}{\kelvin}}^T \frac{7R}{2T} dT + \int_{\SI{1}{\atm}}^P \frac{R}{P} dP.\]
    \begin{equation}
        S = S_{298}^\circ + R \left( \frac72 \ln\frac{T}{\SI{298}{\kelvin}} + \ln\frac{P}{\SI{1}{\atm}} \right)
        \label{eq:4.14.S}
    \end{equation}

    Plot this function given $S_{298}^\circ = \SI{191.5}{\joule\per\mol\per\kelvin}$
    and $R = \SI{8.314}{\joule\per\mol\per\kelvin}$.  This binary function,
    whose plot is a surface in a 3-D space, is preferably plotted with a
    computer program.
\end{answer}

\begin{problem}
\end{problem}

\begin{problem}
    Evaluate the partial derivative
    \[ \left( \frac{\partial H}{\partial G} \right)_S \]
    in terms of experimental variables.
\end{problem}

\begin{answer}
    Enthalpy has a simple isentropic partial derivative.
    \begin{equation}
       \left( \frac{\partial H}{\partial P} \right)_S = V. \label{eq:HPS}
    \end{equation}

    This is not the case for Gibbs free energy.
    \begin{equation}
        dG = -SdT + VdP. \label{eq:dG}
    \end{equation}

    To obtain isentropic partial derivative, try to replace $dT$ with $dS$.
    Expressing entropy with $S(T, P)$ would produce useful differentials.
    \[
        \left( \frac{\partial S}{\partial T} \right)_P
        = \frac1T \left( \frac{\partial Q}{\partial T} \right)_P
        = \frac{C_P}{T}
    \]
    \[
        \left( \frac{\partial S}{\partial P} \right)_T
        = -\frac{\partial^2 G}{\partial T \partial P}
        = -\left( \frac{\partial V}{\partial T} \right)_P
        = -V \alpha
    \]
    \[ dS = \frac{C_P}{T} dT - V \alpha dP. \]

    Replace $dT$ with $dS$ in \eqref{eq:dG} to find
    $\left( \partial G / \partial P \right)_S$.
    \[ dT = \frac{T}{C_P} dS + \frac{TV\alpha}{C_P} dP \]
    \[ dG = -\frac{TS}{C_P} dS + \left( V - \frac{TSV\alpha}{C_P} \right) dP \]
    \begin{equation}
        \left( \frac{\partial G}{\partial P} \right)_S = V - \frac{TSV\alpha}{C_P}. \label{eq:GPS}
    \end{equation}

    Express partial differential with differential quotient to incorporate
    results from \eqref{eq:HPS} and \eqref{eq:GPS}.
    \[
        \left( \frac{\partial H}{\partial G} \right)_S
        = \left( \frac{\partial H}{\partial P} \right)_S \left( \frac{\partial P}{\partial G} \right)_S
        = \frac{C_P}{C_P - TS\alpha}
    \]
\end{answer}

\begin{problem}
\end{problem}

\begin{problem}
\end{problem}

\begin{problem}
\end{problem}

\begin{problem}
\end{problem}

\begin{problem}
\end{problem}

\begin{remark}
    Do not use the differential
    \[ dG = -SdT + VdP \]
    because we cannot express $V$ in $V(P)$.  The volume depends on both
    temperature and pressure, i.e. $V(T, P)$.
\end{remark}

\begin{answer}
    Gibbs energy is defined as
    \begin{equation}
        G = H - TS. \label{eq:4.14.G}
    \end{equation}

    Gibbs energy of a gas is a function of temperature and pressure.  More
    precisely, $H$ can be expressed with $H(T)$ and $S$ with $S(T, P)$ given
    its standard entropy $S_{298}^\circ$.

    For a diatomic gas, its enthalpy is given as
    \begin{equation}
        H = C_P T = \frac72 RT. \label{eq:4.14.H}
    \end{equation}

    It is somewhat more complicated to calculate entropy.  Consider the
    differential of entropy:
    \begin{align*}
        dS &= \frac{C_P}{T} dT - V\alpha dP \\
        &= \frac{7R}{2T} dT - \frac{R}{P} dP.
    \end{align*}

    Therefore, we can find entropy with integration.
    \[ S(T, P) - S_{298}^\circ = \int_{\SI{298}{\kelvin}}^T \frac{7R}{2T} dT + \int_{\SI{1}{\atm}}^P \frac{R}{P} dP.\]
    \begin{equation}
        S = S_{298}^\circ + R \left( \frac72 \ln\frac{T}{\SI{298}{\kelvin}} + \ln\frac{P}{\SI{1}{\atm}} \right)
        \label{eq:4.14.S}
    \end{equation}

    Combine equation \eqref{eq:4.14.G}, \eqref{eq:4.14.H}, and
    \eqref{eq:4.14.S}:
    \[ G = \frac72 RT - S_{298}^\circ T - RT \left( \frac72 \ln\frac{T}{\SI{298}{\kelvin}} + \ln\frac{P}{\SI{1}{\atm}} \right) \]

    Plot this function given $S_{298}^\circ = \SI{130.57}{\joule\per\mol\per\kelvin}$
    and $R = \SI{8.314}{\joule\per\mol\per\kelvin}$.  This binary function,
    whose plot is a surface in a 3-D space, is preferably plotted with a
    computer program.
\end{answer}
\end{@empty}
