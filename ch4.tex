\begin{@empty}
\section{Thermodynamic variables and relations}
\begin{problem}
    Write out the combined statements of the first and second laws for the
    energy functions, $U = U(S, V)$, $H = H(S, P)$, $F = F(T, V)$ and $G = G(T,
    P)$.  Assume $\delta W'$ is zero:
    \begin{enumerate}
        \item Write out all eight coefficient relations.
        \item Derive all four Maxwell relations.
    \end{enumerate}
    for these equations.
\end{problem}

\begin{answer}
    \begin{enumerate}
        \item According to definitions,
            \begin{align*}
                H &= U + PV \\
                F &= U - TS \\
                G &= H - TS.
            \end{align*}

            By the first law,
            \[ \Delta U = Q + W \]
            \[ W(S, V) = -\int PdV \]
            \begin{equation}
                \Delta U(S, V) = Q -\int PdV. \label{eq:1st}
            \end{equation}

            By the second law,
            \[ \Delta S = \int dS = \int \frac{\delta Q}{T} \]
            \begin{equation}
                Q = \int \delta Q = \int TdS. \label{eq:2nd}
            \end{equation}

            Combining equations \eqref{eq:1st} and \eqref{eq:2nd},
            \[ \Delta U = \int dU = \int TdS - \int PdV \]
            \begin{equation}
                dU = TdS - PdV. \label{eq:dU}
            \end{equation}

            The other differentials are computed with \eqref{eq:dU}.
            \begin{align*}
                dH &= TdS + VdP \\
                dF &= -SdT - PdV \\
                dG &= -SdT + VdP.
            \end{align*}

        \item Maxwell relations can be derived with Schwarz's theorem.
            \[
                \frac{\partial^2 U}{\partial S \partial V}
                = \left( \frac{\partial T}{\partial V} \right)_S
                = -\left( \frac{\partial P}{\partial S} \right)_V
            \]
            \[
                \frac{\partial^2 H}{\partial S \partial P}
                = \left( \frac{\partial T}{\partial P} \right)_S
                = \left( \frac{\partial V}{\partial S} \right)_P
            \]
            \[
                -\frac{\partial^2 F}{\partial T \partial V}
                = \left( \frac{\partial S}{\partial V} \right)_T
                = \left( \frac{\partial P}{\partial T} \right)_V
            \]
            \[
                \frac{\partial^2 G}{\partial T \partial P}
                = -\left( \frac{\partial S}{\partial P} \right)_T
                = \left( \frac{\partial V}{\partial T} \right)_P
            \]
    \end{enumerate}
\end{answer}

\begin{problem}
    Derive the ratio relation Equation 4.30:
    \[
        \left( \frac{\partial Z}{\partial X} \right)_Y
        \left( \frac{\partial X}{\partial Y} \right)_Z
        \left( \frac{\partial Y}{\partial Z} \right)_X
        = -1.
    \]
\end{problem}

\begin{answer}
    Let $F$ be a function satisfying
    \[ F(X, Y, Z) = 0.\]

    Applicability of inverse function theorem is assumed in thermodynamic
    setting: $F$ is deemed differentiable and its partial derivatives are
    nonzero.  For brevity, let
    \begin{align*}
        A &= \frac{\partial F}{\partial X} \\
        B &= \frac{\partial F}{\partial Y} \\
        C &= \frac{\partial F}{\partial Z}.
    \end{align*}

    Then the total differential of $F$ is
    \[ dF = AdX + BdY + CdZ = 0.\]

    The needed partial derivatives can be reconstructed from the total
    differential.
    \begin{align*}
        -CdZ &= AdX + BdY \\
        -AdX &= BdY + CdZ \\
        -BdY &= CdZ + AdX
    \end{align*}

    \[
        \left( \frac{\partial Z}{\partial X} \right)_Y
        \left( \frac{\partial X}{\partial Y} \right)_Z
        \left( \frac{\partial Y}{\partial Z} \right)_X
        = \left( -\frac AC \right)
          \left( -\frac BA \right)
          \left( -\frac CB \right)
        = -1.
    \]
\end{answer}

\begin{problem}
    The molar volume of Al$_2$O$_3$ at 25\si\celsius\ and 1 atm is 25.715
    cc/mol.  Its coefficient of thermal expansion is \SI{26e-6}{\per\kelvin}
    and the coefficient of compressibility is \SI{8.0e-7}{\per\atm}.  Estimate
    the molar volume of Al$_2$O$_3$ at 400\si\celsius\ and 10 kbars pressure
    (\SI{10e3}{\atm}).
\end{problem}

\begin{answer}
    The molar volume $V$ is a function of temperature $T$ and pressure $P$.
    This is an initial value problem.
    \[ V(25, 1) = 25.715 \]
    \[ \frac1V \frac{\partial V}{\partial T} = \alpha = \num{26e-6} \]
    \[ -\frac1V \frac{\partial V}{\partial P} = \beta = \num{8.0e-7}.\]

    Convert the system of partial differential equations into differentials.
    \[ dV = V \alpha dT - V \beta dP.\]

    This is a separable equation, which is solvable by mere integration.
    \begin{equation}
        \int \frac1V dV = \int \alpha dT - \int \beta dP. \label{eq:VTP}
    \end{equation}

    Integrate from the initial state.
    \[ \ln \frac{V}{25.715} = \alpha \left( T - 25 \right) - \beta \left( P - 1 \right) \]
    \[ V = 25.715 \exp \Left( \alpha \left( T - 25 \right) - \beta \left( P - 1 \right) \Right).\]

    Compute the volume in the final state with the just solved $V(T, P)$.
    \begin{align*}
        V(400, \num{10e3})
        &= 25.715 \exp \Left(
            \alpha \left( 400 - 25 \right)
            - \beta \left( \num{10e3} - 1 \right) \Right) \\
        &= 25.760 \left( \si{cc/\mol} \right)
    \end{align*}
\end{answer}

\begin{problem}
\end{problem}

\begin{answer}
    \begin{enumerate}
        \item Entropy can be computed with little derivation from its
            definition.
            \[ \Delta S = \int \frac{\delta Q}{T} = \int \frac{C_P}{T} dT \]

            Isobaric specific heat capacity of selected elements can be found
            in Appendix B.
            \[ C_P = 11.17 + \num{37.78e-3}\,T + \frac{\num{3.18e5}}{T^2} \left( \si{\joule\per\mol\per\kelvin} \right) \]

            Heat capacity has the same dimension as entropy.
            \begin{align*}
                \Delta S &= \int_{300}^{1300} \left( \frac{11.17}{T} + \num{37.78e-3} + \frac{\num{3.18e5}}{T^3} \right) dT \\
                    &= \left[ 11.17 \ln T + \num{37.78e-3}\, T - \frac{\num{1.59e5}}{T^2} \right]_{300}^{1300} \\
                    &= 11.17 \ln \frac{1300}{300} + 37.78
                        + \num{1.59e5} \left( \frac{1}{300^2} - \frac{1}{1300^2} \right) \\
                    &\approx 55.83 \left( \si{\joule\per\mol\per\kelvin} \right)
            \end{align*}

        \item Computing isothermal entropy is somewhat tricky because
            involvement of a Maxwell relation.
            \[
                \left( \frac{\partial S}{\partial P} \right)_T
                = -\frac{\partial^2 G}{\partial T \partial P}
                = -\left( \frac{\partial V}{\partial T} \right)_P
                = -V \alpha
            \]
            \[ \Delta S = \int \left( \frac{\partial S}{\partial P} \right)_T dP = -\int V \alpha dP \]

            Solving $V(T, P)$ is omitted because already done in
            \eqref{eq:VTP}.
            \[
                V(T, P) = V^{\text S} \exp \Left(
                    \alpha \left( T - 298 \right)
                    - \beta \left( P - 1 \right) \Right).
            \]
            \[
                V(300, P) = V^{\text S} \exp \Left(
                    2 \alpha
                    - \beta \left( P - 1 \right) \Right).
            \]
            \begin{align*}
                \Delta S &= -V^{\text S} \alpha \int_1^{\num{100e3}}
                    \exp \Left( 2 \alpha - \beta \left( P - 1 \right) \Right) dP \\
                &= \frac{V^{\text S} \alpha}{\beta}
                    \left[ \exp \Left( 2 \alpha - \beta \left( P - 1 \right) \Right) \right]_1^{\num{100e3}}
            \end{align*}

            Note that the equation above would produce energy in \si{\atm.cc}
            if unadjusted due to units for $V^{\text S}$ and $\beta$.  Multiply
            the conversion factor to get energy in joules.
            \begin{center}
                \SI{1}{\atm.cc} = \SI{0.101325}{\joule}
            \end{center}
    \end{enumerate}
\end{answer}

\begin{problem}
\end{problem}

\begin{problem}
\end{problem}

\begin{problem}
\end{problem}

\begin{problem}
\end{problem}

\begin{problem}
    Evaluate the partial derivative
    \[ \left( \frac{\partial H}{\partial G} \right)_S \]
    in terms of experimental variables.
\end{problem}

\begin{answer}
    Enthalpy has a simple isentropic partial derivative.
    \begin{equation}
       \left( \frac{\partial H}{\partial P} \right)_S = V. \label{eq:HPS}
    \end{equation}

    This is not the case for Gibbs free energy.
    \begin{equation}
        dG = -SdT + VdP. \label{eq:dG}
    \end{equation}

    To obtain isentropic partial derivative, try to replace $dT$ with $dS$.
    Expressing entropy with $S(T, P)$ would produce useful differentials.
    \[
        \left( \frac{\partial S}{\partial T} \right)_P
        = \frac1T \left( \frac{\partial Q}{\partial T} \right)_P
        = \frac{C_P}{T}
    \]
    \[
        \left( \frac{\partial S}{\partial P} \right)_T
        = -\frac{\partial^2 G}{\partial T \partial P}
        = -\left( \frac{\partial V}{\partial T} \right)_P
        = -V \alpha
    \]
    \[ dS = \frac{C_P}{T} dT - V \alpha dP. \]

    Replace $dT$ with $dS$ in \eqref{eq:dG} to find
    $\left( \partial G / \partial P \right)_S$.
    \[ dT = \frac{T}{C_P} dS + \frac{TV\alpha}{C_P} dP \]
    \[ dG = -\frac{TS}{C_P} dS + \left( V - \frac{TSV\alpha}{C_P} \right) dP \]
    \begin{equation}
        \left( \frac{\partial G}{\partial P} \right)_S = V - \frac{TSV\alpha}{C_P}. \label{eq:GPS}
    \end{equation}

    Express partial differential with differential quotient to incorporate
    results from \eqref{eq:HPS} and \eqref{eq:GPS}.
    \[
        \left( \frac{\partial H}{\partial G} \right)_S
        = \left( \frac{\partial H}{\partial P} \right)_S \left( \frac{\partial P}{\partial G} \right)_S
        = \frac{C_P}{C_P - TS\alpha}
    \]
\end{answer}
\end{@empty}
