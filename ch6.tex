\begin{@empty}
\section{Statistical thermodynamics}
\begin{problem}
\end{problem}

\begin{problem}
\end{problem}

\begin{problem}
\end{problem}

\begin{answer}
    \begin{enumerate}
        \item There are \( 4^2 = 16 \) microstates.
        \item The microstates are
            \( \left< \varepsilon_1, \varepsilon_1 \right> \),
            \( \left< \varepsilon_1, \varepsilon_2 \right> \),
            \( \left< \varepsilon_1, \varepsilon_3 \right> \),
            \( \left< \varepsilon_1, \varepsilon_4 \right> \),
            \( \left< \varepsilon_2, \varepsilon_1 \right> \),
            \( \left< \varepsilon_2, \varepsilon_2 \right> \),
            \( \left< \varepsilon_2, \varepsilon_3 \right> \),
            \( \left< \varepsilon_2, \varepsilon_4 \right> \),
            \( \left< \varepsilon_3, \varepsilon_1 \right> \),
            \( \left< \varepsilon_3, \varepsilon_2 \right> \),
            \( \left< \varepsilon_3, \varepsilon_3 \right> \),
            \( \left< \varepsilon_3, \varepsilon_4 \right> \),
            \( \left< \varepsilon_4, \varepsilon_1 \right> \),
            \( \left< \varepsilon_4, \varepsilon_2 \right> \),
            \( \left< \varepsilon_4, \varepsilon_3 \right> \),
            and \( \left< \varepsilon_4, \varepsilon_4 \right> \).
        \item The macrostates are
            \( \left\{ \varepsilon_1, \varepsilon_1 \right\} \),
            \( \left\{ \varepsilon_1, \varepsilon_2 \right\} \),
            \( \left\{ \varepsilon_1, \varepsilon_3 \right\} \),
            \( \left\{ \varepsilon_1, \varepsilon_4 \right\} \),
            \( \left\{ \varepsilon_2, \varepsilon_2 \right\} \),
            \( \left\{ \varepsilon_2, \varepsilon_3 \right\} \),
            \( \left\{ \varepsilon_2, \varepsilon_4 \right\} \),
            \( \left\{ \varepsilon_3, \varepsilon_3 \right\} \),
            \( \left\{ \varepsilon_3, \varepsilon_4 \right\} \),
            and \( \left\{ \varepsilon_4, \varepsilon_4 \right\} \).
    \end{enumerate}
\end{answer}

\begin{problem}
\end{problem}

\begin{problem}
\end{problem}

\begin{answer}
    \begin{enumerate}
        \item
            \[ 1\cdot2 + 2\cdot3 + 4\cdot4 + 2\cdot5 + 1\cdot6 = 40 \]
            \[ 1\cdot1 + 1\cdot2 + 2\cdot3 + 2\cdot4 + 2\cdot5 + 1\cdot6 + 1\cdot7 = 40 \]
            These states has the same energy.
        \item
            \[ \Omega_1 = \frac{10!}{2!4!2!} \]
            \[ \Omega_2 = \frac{10!}{2!2!2!} \]
            \[ \frac{\Omega_2}{\Omega_1} = \frac{2!4!2!}{2!2!2!} = 12 \]
            State II has higher entropy.
        \item State II is more likely to be observed.
    \end{enumerate}
\end{answer}

\begin{problem}
\end{problem}

\begin{problem}
\end{problem}

\begin{answer}
    The initial state is
    \[ \left\{ 14, 18, 27, 38, 51, 78, 67, 54, 32, 27, 23, 20, 19, 17, 15 \right\}.\]
    The final state is
    \[ \left\{ 14, 18, 26, 37, 49, 78, 68, 55, 34, 29, 24, 20, 18, 16, 14 \right\}.\]

    Therefore, the difference in entropy is
    \begin{align*}
        \Delta S &= R \ln \frac{\Omega_1}{\Omega_0} \\
        &= R \ln\frac{14!18!27!38!51!78!67!54!32!27!23!20!19!17!15!}
        {14!18!26!37!49!78!68!55!34!29!24!20!18!16!14!} \\
        &= \SI{-15.50}{\joule/\mol.\kelvin} \\
        &= \SI{-2.574e-23}{\joule/\kelvin}.
    \end{align*}
\end{answer}
\end{@empty}
